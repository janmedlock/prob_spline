\documentclass{article}

\usepackage{geometry}

\usepackage{amsmath}

\usepackage{bm}

\usepackage{hyperref}
\hypersetup{breaklinks}
\hypersetup{pdfborder=0 0 0}

\usepackage[terseinits=true, giveninits=true]{biblatex}
\DeclareNameAlias{default}{last-first}
\renewcommand{\revsdnamepunct}{}
\DeclareFieldFormat{title}{\textit{#1}}
\AtEveryBibitem{\ifentrytype{book}{\clearfield{series}}{}}
\addbibresource{notes.bib}

\renewcommand{\vec}[1]{\bm{#1}}
\newcommand{\mat}[1]{\bm{#1}}
\newcommand{\md}{\mathrm{d}}
\DeclareMathOperator{\diag}{diag}


\title{Notes on B-splines}
\author{Jan Medlock}


\begin{document}

\maketitle

Following \textcite{Dierckx_1993}, a B-spline on $x \in [a, b]$ is
\begin{equation}
  s_{k, \vec{\lambda}}(x)
  = \sum_{i=0}^{g+k} c_i N_{i, k, \vec{\lambda}}(x).
\end{equation}

\begin{description}
\item [Degree] $k \geq 0$.

\item [Knots]
  $\vec{\lambda} = [\lambda_0, \ldots, \lambda_{g+k+1}]$,
  with $\lambda_k = a$ and $\lambda_{g+k+1} = b$.

  \begin{itemize}
  \item There are $g + k + 2$ knots.

  \item The knots are non-decreasing:
    $\lambda_i \leq \lambda_{i+1}$.

  \item There are $g$ internal knots:
    $[\lambda_{k+1}, \ldots, \lambda_{g+k}]$.

  \end{itemize}

\item[Continuity] The first $k - 1$ derivatives of the spline
  $s_{k, \vec{\lambda}}(x)$ are continuous on $[a, b]$, except at
  repeated knots.  At a repeated knot
  $\lambda_{i - 1}
  < \lambda_i = \ldots = \lambda_{i + j}
  < \lambda_{i + j + 1}$,
  only the first $k - j - 1$ derivatives of
  $s_{k, \vec{\lambda}}(x)$
  are continuous, i.e.~derivatives
  $k - j - 1, \ldots, k - 1$
  are discontinuous.

\item[Truncated power function]
  \begin{equation}
    (t - x)_+^k
    =
    \begin{cases}
      (t - x)^k & \text{if $t \geq x$},
      \\
      0 & \text{if $t < x$}.
    \end{cases}
  \end{equation}

  \begin{itemize}
  \item For $k = 0$, define
    \begin{equation}
      (t - x)_+^0
      =
      \begin{cases}
        1 & \text{if $t \geq x$},
        \\
        0 & \text{if $t < x$}.
      \end{cases}
    \end{equation}

  \end{itemize}

\item[Divided difference] The $k$th divided difference of the
  function $f$ at the points $\tau_0, \ldots, \tau_k$
  is the leading coefficient of the degree-$k$ polynomial $p_k$ that
  interpolates $f$.  It is denoted
  $\Delta_t^k (\tau_0, \ldots, \tau_k) f(t)$ and
  $[\tau_0, \ldots, \tau_k] f$.

  \begin{itemize}
  \item $p_k(\tau_i) = f(\tau_i)$.

  \item In addition, at points where
    $\tau_i = \ldots = \tau_{i+j}$,
    \begin{equation}
      \frac{\md^{\ell} p_k}{\md x^{\ell}}(\tau_i)
      = \frac{\md^{\ell} f}{\md x^{\ell}}(\tau_i)
      \quad \text{for $\ell = 1, \ldots, j$}.
    \end{equation}

  \item Recursion:
    \begin{equation}
      \begin{split}
        [\tau_i] f &= f(\tau_i),
        \\
        [\tau_i, \ldots, \tau_{i+j}] f
        &=
        \begin{cases}
          \frac{[\tau_{i+1}, \ldots, \tau_{i+j}] f
            - [\tau_i, \ldots, \tau_{i+j-1}] f}
          {\tau_{i+j} - \tau_i}
          & \text{if $\tau_i < \tau_{i + 1}$},
          \\
          \frac{1}{j!} \frac{\md^j f}{\md x^j}(\tau_i)
          & \text{if $\tau_i = \ldots = \tau_{i+j}$}.
        \end{cases}
      \end{split}
    \end{equation}

  \item The polynomial interpolant is
    \begin{equation}
      p_k(x) = [\tau_0] f
      + \sum_{j=1}^k (x - \tau_0) \cdots (x - \tau_{j-1})
      [\tau_0, \ldots, \tau_j] f.
    \end{equation}

  \item The leading coefficient is
    \begin{equation}
      [\tau_0, \ldots, \tau_k] f
      = \frac{1}{k!} \frac{\md^k p_k}{\md x^k}.
    \end{equation}

  \end{itemize}

\item[B-spline basis functions]
  \begin{equation}
    N_{i, k, \vec{\lambda}}(x) = (\lambda_{i+k+1} - \lambda_i)
    \Delta_t^{k+1}(\lambda_i, \ldots, \lambda_{i+k+1}) (t - x)_{+}^k.
  \end{equation}

  \begin{itemize}

  \item Recursion:
    \begin{equation}
      \begin{split}
        N_{i, 0, \vec{\lambda}}(x)
        &=
        \begin{cases}
          \chi_{[\lambda_i, \lambda_{i+1})}(x)
          & \text{if $i < g + k$},
          \\
          \chi_{[\lambda_i, \lambda_{i+1}]}(x)
          & \text{if $i = g + k$},
        \end{cases}
        \\
        N_{i, k, \vec{\lambda}}(x)
        &=
        \frac{x - \lambda_i}{\lambda_{i+k} - \lambda_i}
        N_{i, k-1, \vec{\lambda}}(x)
        + \frac{\lambda_{i+k+1} - x}{\lambda_{i+k+1} - \lambda_{i+1}}
        N_{i+1, k-1, \vec{\lambda}}(x),
      \end{split}
    \end{equation}
    for $i = 0, \ldots, g+k$, where
    \begin{equation}
      \label{eq:3}
      \chi_{\mathcal{S}}(x)
      =
      \begin{cases}
        1 & \text{if $x \in \mathcal{S}$},
        \\
        0 & \text{if $x \notin \mathcal{S}$},
      \end{cases}
    \end{equation}
    is the characteristic function.
    Note that without the modified definition for
    $N_{g + k, 0, \vec{\lambda}}(x)$,
    at the right endpoint,
    $N_{i, k, \vec{\lambda}}(b) = 0$ for all $i$ and $k$.

  \item Derivative:
    \begin{equation}
      \frac{\md N_{i, k, \vec{\lambda}}}{\md x}(x)
      =
      k \left[
        \frac{N_{i, k-1, \vec{\lambda}}(x)}
        {\lambda_{i+k} - \lambda_i}
        - \frac{N_{i+1, k-1, \vec{\lambda}}(x)}
        {\lambda_{i+k+1} - \lambda_{i+1}}
      \right].
    \end{equation}

  \item $N_{i, k, \vec{\lambda}}(x) \geq 0$
    for all $x$.

  \end{itemize}

\end{description}

\section{scipy.interpolate and FITPACK}

The $n \geq k + 1$ data points to fit are $(x_j, y_j)$
for $j = 0, \ldots, n - 1$.

The $n + k + 1$ knots are
\begin{equation}
  \begin{split}
    \lambda_0 &= \cdots = \lambda_k
    = x_0,
    \\
    \lambda_j
    &=
    \left\{
      \begin{array}{ll}
        x_{j - \frac{k+1}{2}}
        & \text{if $k$ is odd}
        \\
        \frac{1}{2}
        \left(
          x_{j - \frac{k}{2} - 1} + x_{j - \frac{k}{2}}
        \right)
        & \text{if $k$ is even}
      \end{array}
    \right\}
    \;
    \text{for $j = k+1, \ldots, n-1$},
    \\
    \lambda_n &= \cdots = \lambda_{n + k} = x_{n-1}.
  \end{split}
\end{equation}
(The middle terms are omitted if $n = k + 1$.)

\emph{What do the locations of the boundary knots imply?}

The coefficients are $\vec{c} = [c_0, \ldots, c_{n-1}]$.  These
are padded to length $n + k + 1$ by appending $k + 1$ zeros.


\subsection{Interpolation}

For interpolating splines,
\begin{equation}
  \begin{split}
    \vec{y}
    &= s_{k, \vec{\lambda}}(\vec{x})
    \\
    &= \sum_{i=0}^{n-1} c_i
    N_{i, k, \vec{\lambda}}(\vec{x}).
    \\
    &= \mat{A} \vec{c},
  \end{split}
\end{equation}
with $\mat{A}$ the matrix with columns
\begin{equation}
  \vec{a}_i
  =
  N_{i, k, \vec{\lambda}}(\vec{x}).
\end{equation}
Only the middle $2 k - 1$ diagonals of $\mat{A}$ are nonzero, so
\begin{equation}
  \mat{A}
  =
  \sum_{i = -(k-1)}^{k-1}
  \diag(\vec{d}_i, i),
\end{equation}
with
\begin{equation}
  \begin{split}
    d_{i, j}
    &=
    N_{i + j, k, \vec{\lambda}}(x_j)
    \quad
    \text{for $j = - \min\{i, 0\}, \ldots, n - \max\{i, 0\} - 1$},
    \\
    &=
    N_{j, k, \vec{\lambda}}(x_{j - i})
    \quad
    \text{for $j = \max\{i, 0\}, \ldots, n + \min\{i, 0\} - 1$}.
  \end{split}
\end{equation}

\section{Variation}

We require
\begin{equation}
  V
  = \int_a^b \left|\frac{\md^2 s_{k, \vec{\lambda}}}{\md x^2}\right|
  \md x.
\end{equation}

The first derivative is
\begin{equation}
  \begin{split}
    \frac{\md s_{k, \vec{\lambda}}}{\md x}
    &= \frac{\md}{\md x} \sum_{i=0}^{n-1} c_i
    N_{i, k, \vec{\lambda}}(x)
    \\
    &= k \sum_{i=1}^{n-1} c_i^{(1)} N_{i, k-1, \vec{\lambda}}(x),
  \end{split}
\end{equation}
with
\begin{equation}
  c_i^{(1)} = \frac{c_i - c_{i-1}}{\lambda_{i+k} - \lambda_i}.
\end{equation}
Then the second derivative is
\begin{equation}
  \begin{split}
    \frac{\md^2 s_{k, \vec{\lambda}}}{\md x^2}
    &= \frac{\md}{\md x} \sum_{i=1}^{n-1} c_i^{(1)}
    N_{i, k-1, \vec{\lambda}}(x)
    \\
    &= k (k - 1) \sum_{i=2}^{n-1} c_i^{(2)}
    N_{i, k-2, \vec{\lambda}}(x),
  \end{split}
\end{equation}
with
\begin{equation}
  c_i^{(2)} =
  \frac{c_i^{(1)} - c_{i-1}^{(1)}}
  {\lambda_{i+k-1} - \lambda_i}.
\end{equation}


\printbibliography

\end{document}
